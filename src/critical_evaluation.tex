\chapter{Critical Evaluation}
\label{chap:evaluation}

% A topic-specific chapter, of roughly 15 pages

% This chapter is intended to evaluate what you did. 
% The content is highly topic-specific, but for many projects will have flavours of the following:

% functional  testing, including analysis and explanation of failure cases,
% behavioural testing, often including analysis of any results that draw some form of conclusion wrt. the aims and objectives, and
% evaluation of options and decisions within the project and/or comparison with alternatives.

\This chapter often acts to differentiate project quality:
even if the work completed is of a high technical quality, critical yet objective evaluation and comparison of the outcomes is crucial.

In essence, the reader wants to learn something,
so the worst examples amount to simple statements of fact
(e.g., ``graph X shows the result is Y'');
the best examples are analytical  and exploratory
(e.g., ``graph X shows the result is Y, which means Z; this contradicts [1], which may be because I use a different assumption'').
As such, both positive {\em and} negative outcomes are valid
{\em if} presented  in a suitable manner.







\section{User Interviews}
why - more data than surveys

ethics

\subsection{Participants}
Each P's time spent


\section{Thematic Analysis}
what it is

coding process

what themes came out


section per theme



\section{Hypotheses}


\begin{enumerate}
    \item Augmenting a climbing session with a live-feed of data analytics will positively impact climbing technique.
    \item A lightweight and simple-to-use product will be popular among intermediate climbers who are serious enough to want to improve, but not so serious they want to pay for coaching.
    \item Seeing a ``score" that rates climbing technique will enable gamification and fun, both for individuals and within groups.
    \item Providing more data to climbers will enable more focused progression tracking and goal-oriented training.
\end{enumerate}
