\chapter{Critical Evaluation}
\label{chap:evaluation}

% A topic-specific chapter, of roughly 15 pages

% This chapter is intended to evaluate what you did. 
% The content is highly topic-specific, but for many projects will have flavours of the following:

% functional  testing, including analysis and explanation of failure cases,
% behavioural testing, often including analysis of any results that draw some form of conclusion wrt. the aims and objectives, and
% evaluation of options and decisions within the project and/or comparison with alternatives.

% This chapter often acts to differentiate project quality:
% even if the work completed is of a high technical quality, critical yet objective evaluation and comparison of the outcomes is crucial.

% In essence, the reader wants to learn something,
% so the worst examples amount to simple statements of fact
% (e.g., ``graph X shows the result is Y'');
% the best examples are analytical  and exploratory
% (e.g., ``graph X shows the result is Y, which means Z; this contradicts [1], which may be because I use a different assumption'').
% As such, both positive {\em and} negative outcomes are valid
% {\em if} presented  in a suitable manner.


Although the iterative design process can be seen as a never-ending spiral that continually refines a product, for the purpose of this project, I stopped iterating and deployed a "final" version of the app.
This allowed me to spend the final two weeks assessing and evaluating, through a series of semi-structured interviews, both how well the app meets the original aims I set out to achieve as a product, and also to what extent my HCI-based hypotheses were accurate.

\section{Format of Final Evaluative Testing}
\subsection{Qualitative vs Quantitative}
Some of the previous research in the field has analysed the efficacy of their climbing products in a quantitative way, by statistically proving predictions of competition placements~\cite{climbaxstudy} or viewing an increase in climbing ability~\cite{climbbsn}. 
However, in the field of HCI, where usability and complex interactions between humans and technology are being examined, it is hard to extract this and reduce the data to numbers than can be statistically tested, leading to more qualitative methods being used~\cite{oro11911}.
Also, for the purposes of the project, recording progression for a quantitative analysis would require much more time than is available, and the hypotheses laid out at the start of the project require the deeper understanding that only qualitative analysis can provide.

\subsection{Thematic Analysis of Interview Data}
\subsubsection{Why this method}
Surveys were a useful source of information for both my initial requirements-gathering and for ongoing feedback points throughout the course of the development, but the static format inhibits elaboration (which leads to richer data), even with the most open-ended of questions~\cite{ozoksurvey}.

Therefore I opted to undertake a series of semi-structured interviews with a range of different users of the augKlimb app, and conduct a Thematic Analysis (TA) over the transcripts, a popular technique in recent HCI work~\cite{themanbrown}.

Semi-structured interviews are open discussions, guided by a set of rough topics or questions, but which allow for the exploration of new ideas by following any leads that come up throughout the conversation, producing rich data~\cite{oro11911}.

TA is a process of encoding qualitative information, and has been presented as the bridge between qualitative and quantitative methods~\cite{boyatzis1998transforming}
Thematic Analysis was developed in the field of psychology, and the most commonly cited methodology I found in HCI literature was the one laid out by Braun and Clarke~\cite{braunclarke06}.
They later wrote a chapter detailing how to apply thematic analysis to interview data, and so a later chapter~\cite{brauminterviewta} written by the same authors, detailing how to apply TA to interview data, was the guide I followed whilst performing the below analysis.

\subsection{Braun \& Clarke’s six phase approach to TA}
Here is the outline of the TA approach, detailed by Braun \& Clarke, that I performed:

\begin{quote}
\begin{enumerate}
    \item \textbf{Familiarisation with the data:} reading and re-reading the data.
    \item \textbf{Coding:} generating succinct labels that identify important features of the data relevant to answering the research question; after coding the entire dataset, collating codes and relevant data extracts.
    \item \textbf{Searching for themes:} examining the codes and collated data to identify significant broader patterns of meaning; collating data relevant to each candidate theme.
    \item \textbf{Reviewing themes:} checking the candidate themes against the dataset, to determine that they tell a convincing story that answers the research question. Themes may be refined, split, combined, or discarded.
    \item \textbf{Defining and naming themes:} developing a detailed analysis of each theme; choosing an informative name for each theme.
    \item \textbf{Writing up:} weaving together the analytic narrative and data extracts; contextualising the analysis in relation to existing literature.
\end{enumerate}

\hspace*{\fill}(Direct quote from~\cite{brauminterviewta})
\end{quote}

\section{Performing the Interviews}
\subsection{Question Selection}
The goals of my interviews were two-fold: to both generally assess the usability and usefulness of the app, and also assess my hypotheses.
With this in mind, I developed a list of nine questions, which started by asking the climbers about their personal climbing (to both provide context and get the conversation flowing), and then lead on to how the app impacted their climbing, and what their thoughts were about using it.

\begin{itemize}
    \item What do your climbing sessions usually include?
    \item Would you class your sessions as fun, training, or a mixture?
    \item How do you think using the app impacted your climbing session today?
    \item Do you see the app as a training tool or as gamification of (adding fun to) your climbs, or both?
    \item Do you think the app caused or helped improve your climbing?
    \item What was your favourite aspect?
    \item What was your least favourite aspect?
    \item Are you likely to continue using the app in the future?
    \item What feature(s) would you like to see extended or added?
\end{itemize}

These questions were only used as a guideline in the interviews, with more questions being asked to explore interesting points that were raised  throughout the conversation.
Although I aimed to obtain answers to all the questions, the wording and order of the questions varied, and often a participant would cover a question without prompting, during discussion leading from another question.


\subsection{Ethics}
Because this was a distinct user-study, with different aims and data-collection methods than my initial study, I applied for a second full ethics approval, the paperwork for which can be found in Appendix \ref{appx:ethics2}.
Privacy and anonymity concerns related to the recording of audio was the biggest challenge.
This was solved by transcribing the recorded audio files to text, before deleting the original recordings.
A secure list of names was kept to enable the particpants' right to withdraw, but any personally identifiable data was removed from the transcripts, and each participant was assigned a number, which is how I will refer to them throughout the below analysis.

\subsection{Participants}
I interviewed two males and four females, five of which were university students, and all were between the ages of 20 and 30.
It should be noted that this is not the most representative sample of the general climbing population, especially with regard to age, technology usage and climbing experience; many climbers are older and have been climbing for multiple decades.

\subsection{Transcription}
As stated above, after recording the interviews, I transcribed the audio into text, to enable the TA to be performed on the data.
Due to both privacy concerns and the inaccuracies of speech-recognition software, I manually transcribed the forty minutes of audio.
The full transcriptions from all six interviews can be found in Appendix~\ref{appx:transcriptions}.

Although this process was painstakingly laborious, and took over eight hours of typing to perform, it did also help to fulfil the first step of TA: to familiarise myself with the data.


\section{Thematic Analysis}
\subsection{Familiarisation}
Transcribing the audio helped begin my familiarisation with the data, but as this was a mostly passive process, trying to type the words as quickly as I could hear them, it did not provide me with the analytical immersion required for TA.
Therefore I repeatedly read through the transcripts, actively thinking about how the text applied to my research goals, and ``treating the data as \textit{data}"\cite{brauminterviewta} until I was fully engaged with the concepts highlighted through the discussion

\subsection{Coding}
Leading on from the familiarisation, I began to systematically highlight and annotate the ideas and interesting points being raised by my interviewees. 
Where similarities arose, I ensured that the same annotation was applied consistently: these annotations were then \textit{codes}.

Multiple passes through the dataset was required as my analysis gradually became more developed, and more nuanced points were highlighted and re-discovered in previously annotated data.


\subsection{Theme Identification}
Patterns gradually arose in the codes I was noting down, also known as ``themes" in the TA terminology. 
I kept these themes in mind as I read through the data set again, colour-coding codes that fell within the various meanings.
After reviewing the themes, and the various codes that they collated, against the transcripts again, I defined and named them.

The themes that arose, and a brief selection of the codes that characterised them, can be found in Table~\ref{tab:themes}.

\begin{table}[h]
\begin{tabular}{|l|l|}
\hline
Route-selection     & 
warming-up with easy climbs, technique focus on easy climbs,
\\& repeating the same climb, attempts to climb at limits, 
\\& app more useful on easy climbs
\\ \hline

Seriousness of a climbing session   &  
fun, training, gamification, social interaction, competition with self,
\\& competition with others
\\ \hline

Complexity of analytics provided    & 
quick performance feedback, detailed analytic feedback,
\\& simple score, graph spikes, video to graph visualisation,
\\& request for labelling technique as "good or bad"
\\ \hline

Ease of use                   & 
easy to use, simple, request for more instructions, 
\\& not wanting to look at the screen, too many button presses,
\\& difficulty in connecting video, gui output is good,
\\& linking repeated attempts of climbs
\\ \hline

Mobile phone as a form-factor       & 
suggestion for wristband, instant display useful, lack of pockets
\\ \hline

\end{tabular}
\label{tab:themes}
\end{table}

These themes align quite closely with both my hypotheses and with the questions asked in the interviews.
There are two reasons for this: the interviews were conducted with guideline questions, so the discussions prompted from these guidelines will naturally follow similar topics, and also I had my research aims in the back of my mind whilst performing both the coding and theme-collation steps, as recommended by the guide I was following~(\cite{brauminterviewta}).


\subsection{Discussion}

\subsubsection{Route-selection}
% warming-up with easy climbs, technique focus on easy climbs,
% repeating the same climb, attempts to climb at limits, 
% app more useful on easy climbs

\subsubsection{Seriousness of a climbing session}   
% fun, training, gamification, social interaction, competition with self,
% competition with others

\subsubsection{Complexity of analytics}
% quick performance feedback, detailed analytic feedback,
% \\& simple score, graph spikes, video to graph visualisation,
% \\& request for labelling technique as "good or bad"

\subsubsection{Ease of use}
% easy to use, simple, request for more instructions, 
% not wanting to look at the screen, too many button presses,
% difficulty in connecting video, gui output is good,
% linking repeated attempts of climbs


\subsubsection{Mobile phone as a form-factor} 
% suggestion for wristband, instant display useful, lack of pockets





\section{Hypotheses}


\begin{enumerate}
    \item Augmenting a climbing session with a live-feed of data analytics will positively impact climbing technique.
    \item A lightweight and simple-to-use product will be popular among intermediate climbers who are serious enough to want to improve, but not so serious they want to pay for coaching.
    \item Seeing a ``score" that rates climbing technique will enable gamification and fun, both for individuals and within groups.
    \item Providing more data to climbers will enable more focused progression tracking and goal-oriented training.
\end{enumerate}
