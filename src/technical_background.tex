\chapter{Technical Background}
\label{chap:technical}

{\bf A compulsory chapter,     of roughly $10$ pages}
\vspace{1cm}

\noindent
This chapter is intended to describe the technical basis on which execution
of the project depends.  The goal is to provide a detailed explanation of
the specific problem at hand, and existing work that is relevant (e.g., an
existing algorithm that you use, alternative solutions proposed, supporting
technologies).

Per the same advice in the handbook, note there is a subtly difference from
this and a full-blown literature review (or survey).  The latter might try
to capture and organise (e.g., categorise somehow) _all_ related work,
potentially offering meta-analysis, whereas here the goal is simply to
ensure the dissertation is self-contained.  Put another way, after reading
this chapter a non-expert reader should have obtained enough background to
understand what *you* have done (by reading subsequent sections), then
accurately assess your work.

You might view an additional goal as giving the reader confidence that you are able to absorb, understand and clearly
communicate highly technical material.



climbing with HCI in general, various types

look up history? timeline

projector stuff

climbing wall - interactive weighted holds? - design concept?


moving onto non-wall stuff, lightweight
climbax study - uses metrics
climbax wristbands

the kickstarter that failed

iphone store apps - apple watch

vertical life - outdoors climb-ticking-off app has started collabs with indoor gyms



trying to develop a middle ground, something that everyone can use without spending money on a specific device, yet still can provide data more meaningful than just logging climbs.

nothing to visualise the climbs, no apps so far use recording or any video-analysis, before developing this app you could use an accelerometer to get some simple data (eg garmin's climbing setting), or get a mate to record you on your phone, but now linking that all together and making a historical record...






non-climbing background
- video analysis for sports
acceleroometer data used elsewhere
mobile phone running apps using pedometer etc



app design research?
eg max-menu item stuff







Various aspects of adding computer-interaction to climbing have been studied in recent years: - using projectors to illuminate and gamify the climbing wall (https://dl.acm.org/citation.cfm?id=2581139) - using wearables to analyse and assess competitors abilities (https://axivity.com/case-studies/climbax) - assisting route-setters (https://www.ncbi.nlm.nih.gov/pubmed/22463006) - manufacturing custom lit holds to add variety to single routes (https://designawards.core77.com/interaction/51284/edge-an-interactive-training-wall-for-climbers) - using computer-vision to analyse the centre-of-mass of climbers, and thus technique https://www.sciencedirect.com/science/article/pii/S0167945707000395
