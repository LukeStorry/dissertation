\chapter{Contextual Background}
\label{chap:context}

{\bf A compulsory chapter,     of roughly $5$ pages}
\vspace{1cm}

\noindent
This chapter should describe the project context, and motivate each of
the proposed aims and objectives.  Ideally, it is written at a fairly
high-level, and easily understood by a reader who is technically
competent but not an expert in the topic itself.

In short, the goal is to answer three questions for the reader.

First, what is the project topic, or problem being investigated?

what is bouldering

indoor sport
training
concepts of grades

"grade levels"
different difficulties etc




Second, why is the topic important, or rather why should the reader care about it?

For example, why there is a need for this project (e.g., lack of similar software or deficiency in existing software), who will benefit from the project and in what way (e.g., end-users, or software developers) what work does the project build on and why is the selected approach either
important and/or interesting (e.g., fills a gap in literature, applies results from another field to a new problem).

why
breaking through plataus
incentivising high-volume low-grade practice
focus on technique
climbing gyms can go for months without resetting routes, so to keep things interesting when repeating the same easy climbs algorithms

Also to see how climbers interact with a more modern product or app,
many climbers "just climb", leaving their phone in the lockers,
or if they do track it they simply list which climbs they have completed in a logbook or logging-app.
Does the adding of some measurements and metrics aid a climber in the long run,
either by making climbing more fun and therefore causing them to go train more often, or does the frequency not change as much as the effectiveness of the training itself - instead of aimlessly climbing with a vague goal of "getting better", does providing quantitative analysis of the climbs completed give more of a focus to the sessions that a climber does?






Finally, what are the central challenges involved and why are they significant?

building a product that meets the requirements and needs of as many climbers/boulderers as possible

Developing an app that can record, share, and match accelerometer data and video recordings taken from a variety of devices at different times

designing a series of metrics that are useful and consistently comparable between climbs

open cv stuff?


running the analysis at the end / throughout?
constant user testing and re-evaluation of goals, quickly developing features in time for a twice-weekly testing session at the wall






The chapter should conclude with a concise bullet point list that
summarises the aims and objectives.  For example:

\begin{quote}
\noindent
The high-level objective of this project is to reduce the performance
gap between hardware and software implementations of modular arithmetic.
More specifically, the concrete aims are:

\begin{enumerate}
\item Research and survey literature on public-key cryptography and
      identify the state of the art in exponentiation algorithms.
\item Improve the state of the art algorithm so that it can be used
      in an effective and flexible way on constrained devices.
\item Implement a framework for describing exponentiation algorithms
      and populate it with suitable examples from the literature on
      an ARM7 platform.
\item Use the framework to perform a study of algorithm performance
      in terms of time and space, and show the proposed improvements
      are worthwhile.
\end{enumerate}
\end{quote}
