\chapter{Conclusion}
\label{chap:conclusion}

{\bf A compulsory chapter,     of roughly $5$ pages} 
\vspace{1cm} 

\noindent
The concluding chapter of a dissertation is often underutilised because it 
is too often left too close to the deadline: it is important to allocation
enough attention.  Ideally, the chapter will consist of three parts:

\begin{enumerate}
\item (Re)summarise the main contributions and achievements, in essence
      summing up the content.
\item Clearly state the current project status (e.g., ``X is working, Y 
      is not'') and evaluate what has been achieved with respect to the 
      initial aims and objectives (e.g., ``I completed aim X outlined 
      previously, the evidence for this is within Chapter Y'').  There 
      is no problem including aims which were not completed, but it is 
      important to evaluate and/or justify why this is the case.
\item Outline any open problems or future plans.  Rather than treat this
      only as an exercise in what you {\em could} have done given more 
      time, try to focus on any unexplored options or interesting outcomes
      (e.g., ``my experiment for X gave counter-intuitive results, this 
      could be because Y and would form an interesting area for further 
      study'' or ``users found feature Z of my software difficult to use,
      which is obvious in hindsight but not during at design stage; to 
      resolve this, I could clearly apply the technique of Smith [7]'').
\end{enumerate}