\chapter{Conclusion}
\label{chap:conclusion}

{\bf A compulsory chapter,     of roughly $5$ pages} 
\vspace{1cm} 

\noindent
The concluding chapter of a dissertation is often underutilised because it 
is too often left too close to the deadline: it is important to allocation
enough attention.  Ideally, the chapter will consist of three parts:

\begin{enumerate}
\item (Re)summarise the main contributions and achievements, in essence
      summing up the content.
\item Clearly state the current project status (e.g., ``X is working, Y 
      is not'') and evaluate what has been achieved with respect to the 
      initial aims and objectives (e.g., ``I completed aim X outlined 
      previously, the evidence for this is within Chapter Y'').  There 
      is no problem including aims which were not completed, but it is 
      important to evaluate and/or justify why this is the case.
\item Outline any open problems or future plans.  Rather than treat this
      only as an exercise in what you {\em could} have done given more 
      time, try to focus on any unexplored options or interesting outcomes
      (e.g., ``my experiment for X gave counter-intuitive results, this 
      could be because Y and would form an interesting area for further 
      study'' or ``users found feature Z of my software difficult to use,
      which is obvious in hindsight but not during at design stage; to 
      resolve this, I could clearly apply the technique of Smith [7]'').
\end{enumerate}





Must Have: At least 2 wizard-of-oz prototypes to test user interaction and preferences.
A final product that implements some of the features at a low-fi level. Some form of testing and analysis of both prototypes and the final product.
Full ethics and health-and-safety approval.


Should Have:
An analysis of the current needs and wants of both developing intermediate-level climbers.
A final product that fully works and is very useful for intended purpose.
Plenty of user testing to give both qualitative and quantitative results for both the prototypes and final design.

Could Have: Strong usage of User-centred-design techniques.
Usage of a very novel technology/technologies as part of the final product.
An excellent writeup analysing choices and mistakes made along the development process.
A final product that is ready for market.
