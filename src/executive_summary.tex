\chapter*{Executive Summary}


% This section should pr\'{e}cis the project context, aims and objectives, and main contributions (e.g., deliverables) and achievements; the same section may be called an abstract elsewhere.
% The goal is to ensure the reader is clear about what the topic is, what you have done within this topic, {\em and} what your view of the outcome is.

% The former aspects should be guided by your specification: essentially this section is a (very) short version of what is typically the first chapter. 

\noindent
Climbing is a popular and growing sport, especially indoors, where climbers can train on man-made routes using artificial holds.
Both strength and good technique is required to successfully reach the top of a climb, and often coaches work to improve technique so less strength is required, enabling a climber to ascent more difficult climbs.
Various aspects of adding computer-interaction to climbing have been studied in recent years, but there is a large space for research into lightweight tools to aid recreational intermediate climbers, both with trickier climbs and to improve their own technique.

In this project, I explored which form of data-capture and output-features could improve a climber's training, and analysed how climbers responded to viewing their data throughout a climbing session, then conducted a user-centred design to build a lightweight mobile application for intermediate climbers.

A variety of hardware and software solutions were explored, tested and developed through  series of surveys, discussions, wizard-of-oz studies and prototyping, resulting in a system that most closely meets the needs of local indoor boulderers given the project's time scope.




\noindent
\begin{itemize}
\item I spent over $60$ hours conducting in-field observations of climbers interacting with various prototypes.
\item I iteratively developed an interactive mobile app that: \begin{itemize}
      \item can record, graph, and score the acceleration of a climber, as both a training tool and gamification incentive for good technique
      \item can link a video recording to the acceleration graph, to enable frame-by-frame inspection of weaknesses
      \item is fully approved and distributed on the Google play Store and currently being regularly used by 15 local climbers.
\end{itemize}
\item I wrote over $1000$ lines of \verb|C#| source code, with a further $20,000$ lines of Unity code-files defining the graphical interface.
\item I conducted a final usability study, comprising a thematic analysis of forty minutes's worth of interview transcripts, to gain a deep understanding of the app's impact on the climbers using it, along with its benefits and limitations. 
\end{itemize}