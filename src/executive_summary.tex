\chapter*{Executive Summary}

{\bf A compulsory section, of at most $1$ page} 
\vspace{1cm} 

\noindent
This section should pr\'{e}cis the project context, aims and objectives, and main contributions (e.g., deliverables) and achievements; the same section may be called an abstract elsewhere.
The goal is to ensure the reader is clear about what the topic is, what you have done within this topic, {\em and} what your view of the outcome is.

The former aspects should be guided by your specification: essentially this section is a (very) short version of what is typically the first chapter. 



Climbing is a popular and growing sport, especially indoors, where climbers can train on man-made routes using artificial holds.
Both strength and good technique is required to successfully reach the top of a climb, and often climbing coaches work to improve technique so less strength is required, yet it is often difficult to see and suggest improvements without years of coaching experience.

There is a space for a tool to aid learning/intermediate climbers, both with trickier climbs and to improve their own technique.


Exploring which form of data-capture and output-features would improve a climber's training?
How do climbers respond to viewing their data throughout a training session?

I propose to conduct a user-centered design to find requirements for, and to build, a lightweight application/product for intermediate climbers. A series of interviews, wizard-of-oz studies, and prototyping will take place, resulting in a system that most closely meets the needs of local indoor boulderers. The hardware is initially planned to be a mixture of: - a mobile phone with an app for computer-vision and outputting results - building electronics into chalkbags (an item of equipment commonly carried around a waist-belt) - Voice-output via bluetooth headphones The specific input and output features, aand the level of gamification or advice given, will be determined as a result of testing.



Must Have: At least 2 wizard-of-oz prototypes to test user interaction and preferences.
A final product that implements some of the features at a low-fi level. Some form of testing and analysis of both prototypes and the final product.
Full ethics and health-and-safety approval.


Should Have:
An analysis of the current needs and wants of both developing intermediate-level climbers.
A final product that fully works and is very useful for intended purpose.
Plenty of user testing to give both qualitative and quantitative results for both the prototypes and final design.

Could Have: Strong usage of User-centred-design techniques.
Usage of a very novel technology/technologies as part of the final product.
An excellent writeup analysing choices and mistakes made along the development process.
A final product that is ready for market.









\noindent
\begin{itemize}
\item I spent over $60$ hours conducting in-field observations of climbers interacting with various prototypes of data-collection and analysis.
\item I iteratively developed an interactive mobile app that: \begin{itemize}
      \item can record, graph, and score the acceleration of a climber, as both a training tool and gamification incentive for good technique
      \item can link a video recording to the acceleration graph, to enable frame-by-frame inspection of weaknesses
      \item is fully approved and distributed on the Google play Store and currently being regularly used by 15 local climbers.
\end{itemize}
\item I wrote over $1000$ lines of \verb|C#| source code, with a further $20,000$ lines of Unity code-files defining the graphical interface.
\item I conducted a final usability study, comprising a thematic analysis of an hour's worth of interview transcripts, to gain a deep understanding of the app's impact, benefits and limitations on the climbers using it. 
\end{itemize}